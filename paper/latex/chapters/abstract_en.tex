\documentclass[class=NCU_thesis, crop=false]{standalone}
\begin{document}

\chapter{Abstract}

The violin has long been a popular instrument for learning and performance, 
with many well-known pieces and distinguished violinists. 
Among these pieces, ensemble compositions involving the violin and other instruments are predominant, 
requiring collaboration with other instrumentalists to fully present the musical piece.
However, due to time or cost constraints, 
finding long-term ensemble partners (accompanists) can be challenging due to time or cost constraints.
Online public resources often provide mixed audio, which does not yield good ensemble effects.
Therefore, this research focuses on developing a system for the common violin and piano ensemble. 
This system can separate the violin and piano sources from a mixed audio source,
track the violin's performance using the separated audio, 
and output the piano accompaniment.

The goal of this research is to develop a real-time music tracking system 
that utilizes source separation results to track violin performances.
We designed a source separation module and a music tracking module. 
For the source separation module, 
we collected and established a new open integrated dataset to train the Band-Split RNN model, 
improving the model's band-split method. 
We evaluated the model using the Signal-to-Distortion Ratio 
to measure the separation performance.
The results show that the model outperforms existing baseline models 
in both data-limit and data-rich cases, 
demonstrating the effectiveness of the band-split method.
For the music tracking module, 
we improved the Online Dynamic Time Warping algorithm and 
the Greedy Backward Alignment method, 
reimplementing the design of the real-time music tracking module and 
enhancing some blocks. 
In practical tests, the real-time music tracking system exhibited 
low latency and accurate tracking performance, 
maintaining stable tracking results comparable to offline tracking 
across different feature tracking performances.

\vspace{2em}
\noindent \textbf{Keywords:} \keywordsEn{} % Set keywords in config.tex
\end{document}