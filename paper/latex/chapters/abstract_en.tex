\documentclass[class=NCU_thesis, crop=false]{standalone}
\begin{document}

\chapter{Abstract}

The violin has long been a popular instrument for learning and performance, 
with many well-known pieces and distinguished violinists. 
Among these pieces, the majority are ensemble works involving the violin and other instruments, 
requiring collaboration to fully present the piece's essence. 
However, due to time or cost constraints, 
finding long-term ensemble partners (accompanists) is not always feasible. 
Thus, this study aims to develop a system specifically 
for the common violin and piano ensemble. 
This system can separate the mixed audio source of the violin and piano, 
track the violin's performance using the separated audio, 
and output the piano accompaniment.

This study aims to develop a violin performance tracking system 
incorporating a source separation model to achieve mixed source separation 
of violin and piano, 
and to utilize the separated sources for music tracking under different features.

We designed a source separation module and a music tracking module. 
For the source separation module, 
we collected and established a new public integrated dataset to train the Band-Split RNN model, 
and improved the model's band-split method. 
In model evaluation, we used the Signal-to-Noise Ratio 
to measure the separation performance.
The results show that the model outperforms existing baseline models 
in both data-limit and data-rich cases, 
demonstrating the effectiveness of the band-split method.

For the music tracking module, 
we improved the Online Dynamic Time Warping algorithm and 
the Greedy Backward Alignment method, 
reimplementing the design of the real-time music tracking module and 
enhancing some blocks. 
In practical tests, the real-time music tracking system exhibited 
low latency and accurate tracking performance, 
maintaining stable tracking effects under different features.

\vspace{2em}
\noindent \textbf{Keywords:} \keywordsEn{} % Set keywords in config.tex
\end{document}