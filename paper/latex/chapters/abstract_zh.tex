\documentclass[class=NCU_thesis, crop=false]{standalone}
\begin{document}

\chapter{摘要}
小提琴一直以來都是許多人學習與演奏的樂器,有許多膾炙人口的曲子與優秀的小提琴音樂家。
在眾多曲子中,小提琴與其他樂器的合奏曲子佔多數,
因此需要其他樂器的演奏者一起合奏才能完整呈現曲子的風貌。
然而,由於時間或成本的因素,尋找長期合作的合奏者(伴奏者)並不是那麼容易,
網路上的公開資源又多為混合音訊,合奏的效果不佳。
因此本研究針對最常見的小提琴與鋼琴的合奏方式來開發一套系統,
此系統可將混合音源中的小提琴與鋼琴音源分離,
並使用分離音源追蹤現場小提琴演奏,輸出鋼琴伴奏。

本研究旨在開發一套使用音源分離結果實現小提琴演奏追蹤的即時音樂追蹤系統,
我們設計了音源分離模組與音樂追蹤模組,
在音源分離模組方面,我們自行蒐集並建立一套新的公開整合資料集,
用於訓練Band-Split RNN模型,並改進了模型的頻帶切割方法。
在模型的評估上,我們使用訊號失真比來計算模型的分離效果,
結果顯示模型在資料缺乏與資料充足的情況下皆優於現有的基線模型,並證明頻帶切割方法的有效性。
在音樂追蹤模組方面,我們改進了線上動態時間規整演算法與貪心向後對齊方法,
重現了即時音樂追蹤模組的設計,並改良部分元件。
在實際的測試中,即時音樂追蹤系統展現了低延遲與精準追蹤的表現,
並在不同特徵的追蹤表現上保持了與離線追蹤相同穩定的追蹤效果。

% 研究背景: 簡述論文的工作為什麼重要,包括對科研背景和工作範圍的論述,以及對整個學科的貢獻意義。
% 研究目的: 簡述論文要解決什麼問題,填補哪些空白。
% 研究方法: 簡述論文如何解決問題,採用什麼理論和實驗方法等。
% 研究發現: 簡述論文的發現即結果,包括新的科學現象或科研方法。
% 研究結論: 簡述結論,並展望未來的研究工作,預期論文的發現對相關領域的影響。

\vspace{2em}
\noindent \textbf{關鍵字:} \keywordsZh{} % Set keywords in config.tex
\end{document}