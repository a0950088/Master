\documentclass[class=NCU_thesis, crop=false]{standalone}
\begin{document}

\chapter{緒論}
\section{研究動機}

根據大學術科考試委員會107年至112年的音樂術科考試人數資料統計~\cite{CAPE2024Statistics},
樂器主修報考最多的項目分別為弦樂與鋼琴,
其中弦樂主修又以小提琴佔比最高。由此可知,小提琴是許多人學習和演奏的樂器。
在眾多涉及小提琴的樂曲中,除了無伴奏小提琴曲(例如巴赫無伴奏小提琴奏鳴曲等)之外,
幾乎都需要與其他樂器合奏,而鋼琴則是最為常見的合奏樂器。
例如小提琴奏鳴曲就是由小提琴與鋼琴共同演奏的曲子,
此外也有許多曲目從原始樂器編制改編為鋼琴與小提琴的合奏版本。

通常演奏合奏曲目,必須自行尋找或是聘請其他演奏者共同演奏,
但聘請其他演奏者的費用並不便宜,
通常一節伴奏(約50分鐘到1小時)會根據專業度的不同來收費,
平均收費約為台幣600元至1800元不等~\cite{2020PianoAccompanistHourlySalary}。
因此若找不到其他演奏者,通常只能選擇演奏自己的部分或是在網路上尋找伴奏音訊。
雖然有些演奏者會將音訊上傳至公開平台提供大家使用,
但這些音訊通常已經是混合音訊,無法根據個人習慣或練習的速度演奏,
也會被音訊中演奏同一部份的聲音干擾。

因此若能將網路上公開的混合音訊分離出伴奏音訊,並使伴奏音訊跟隨自己演奏的速度播放,
便可以在無法找到其他演奏者時,自主練習合奏,同時也省下了人事成本。

近年來,隨著音樂數位化與深度學習的進步,音樂資訊檢索(Music Information Retrieval, MIR)這門領域的發展越來越受到許多人的關注,
這門領域包含音樂來源分離(Music Source Separation, MSS)~\cite{défossez2021music, Cano2019Musical, Rafii2018Overview}、
自動伴奏(Automatic Accompaniment)~\cite{davies2007towards, li2020application, zhang2023design}、
樂譜追蹤(Music Score Tracking)~\cite{orio2003score, dorfer2016towards}、
音樂生成(Music Generation)~\cite{ji2020comprehensive, hernandez2022music}、
樂器辨識(Musical Instrument Recognition)~\cite{solanki2022music, racharla2020predominant, manilow2020hierarchical}等子領域,
這些研究領域也被廣泛的應用於商業化的產品上,例如音樂推薦系統~\cite{Mangla2023Spotify}、
音樂創作工具~\cite{Ableton2024Ableton11, Apple2024LogicPro, PreSonus2024StudioOne}、
音樂教育應用軟體~\cite{Ronimusic2024Amazing, FORSCORE2024forScore}等。

國際音樂資訊檢索協會(International Society for Music Information Retrieval, ISMIR)~\cite{ISMIR2024InternationalSociety}
自2000年開始每年舉行MIR研討會,促進相關領域的交流。
其中音樂來源分離與自動伴奏更是近幾年許多人關注的領域,
音樂來源分離的主要目標為分離混和音訊中的各個音源(樂器),
分離出來的音源可應用於音樂的重新混音,如卡拉OK、DJ混音等,
或作為其他問題的前處理工具~\cite{zhao2022research}。
Sony與ISMIR在2021年舉辦了音樂解混(Music Demixing, MDX)競賽~\cite{Yuki_Mitsufuji2021MusicDemixing},
大力推動了這項技術的發展,近期也延續了先前的成果舉辦更大的聲音分離(Sound Demixing, SDX)競賽~\cite{Fabbro_Giorgio2023TheSoundDemixing}。
自動伴奏的主要目標為根據特定的旋律生成伴奏~\cite{wang2022songdriver, ding2023museflow},
或是追蹤特定的旋律跟隨樂譜~\cite{brazier2021improving}。
這項技術已經在音樂教育與音樂創作~\cite{Antescofo2024metronautapp}等商業產品使用。

在上述的兩個領域中,音樂來源分離的研究重點主要集中在流行樂音源的分離,
自動伴奏使用的音訊資料大部分也是來自可通過MIDI協定傳輸的樂器。
然而古典樂器因為音色的複雜性與資料的稀缺性,在音源分離與自動伴奏領域的研究相對較少,
目前也尚未有將音源分離的結果結合於自動伴奏的研究,
因此本研究旨在開發一套專注於追蹤小提琴演奏的即時音樂追蹤系統,
此系統應用音源分離技術將混合音源分離為參考音訊與伴奏音訊,
並作為即時音樂追蹤系統的參考音訊使用,讓個人練習時有方便的伴奏系統可以使用,
也提供音樂創作者更多不同的創作方式。

% 放第二章?
% 古典樂器的音檔儲存方式多為MP3、WAV等數位音頻資料...(音樂音色的複雜性、資料集的完整性OOOXXX)
% \begingroup
% \parindent0em
% \leftskip2em
% 1. 通過MIDI協定傳輸的資料比數位音頻資料(MP3/WAV)好處理 \\
% 2. \\
% 3. \\
% \par
% \endgroup
% 發展受限
% 音源分離: 樂器大多為較為主流的流行樂樂器
% 節拍追蹤: 古典樂的特徵較為複雜,且除了交響樂等編制較大的樂團,大多時候是沒有打擊樂樂器的,導致節拍追蹤在古典樂器上的應用並不理想。
% 另外古典樂器多為利用樂器本身的外型和設計來發出聲音,因此要將古典樂器的聲音轉為音樂數位介面MIDI來儲存並不容易
% 古典樂器錄音通常為waveform,不會有額外的MIDI資訊
% 因此對於古典樂器通常不會使用MIDI作為系統的輸入資料OOOXXX
% 音源分離的目標都放在人聲、鼓、貝斯等樂器組成的樂團,較少使用在一般古典樂器的音源上blablabla
% 音樂追蹤目前較成熟的應用在樂譜追蹤上較多,只使用樂器音源追蹤的應用較少blablabla
\pagebreak

\section{研究目的}

本研究的目的是開發一套專注於追蹤小提琴演奏的即時音樂追蹤系統,此系統分兩部分研究,
第一部分為音源分離技術的研究,
此研究專注於探討如何分離小提琴與鋼琴的混合音訊,作為後續音樂追蹤的參考音訊使用。
研究詳細內容包含探討如何根據不同樂器的特性調整資料前處理的方式,以提升音源分離模型的效果、
應用深度學習網路訓練音源分離模型並與過往表現突出的模型比較效果。
第二部分為音樂追蹤技術的研究,
此研究專注於探討如何使用來源不同的參考音訊來追蹤現場小提琴演奏的樂曲位置,
並輸出對應的伴奏達到即時合奏的效果。
研究詳細內容包含探討系統行程與線程的設計,平均分配系統計算資源來達到即時的效果、
設計參考音訊與現場串流音訊的特徵提取方式、
設計音樂偵測模組判斷現場演奏是否開始、
改良粗略估計位置模組、線上動態時間規整演算法與貪心向後對齊方法提升追蹤位置的準確率、
設計決策模組決定最後輸出位置並輸出對應的伴奏音訊。

% 專注在分離小提琴與鋼琴的混合音訊並將分離音訊作為音樂追蹤的參考資料,OOOXXX
% 希望能對古典樂器在MIR的領域上有所貢獻OOOXXX

% 目標: 實現使用分離的音源追蹤不同音源的位置並實現即時伴奏
% 建立一套可自行挑選喜歡的合奏來自動追蹤小提琴和鋼琴音源的系統

\pagebreak

\section{論文架構}
本論文分為五個章節,其架構如下:

第一章、緒論,敘述本論文之研究目的、動機以及架構。

第二章、背景知識以及文獻回顧,
介紹本研究所需的背景知識,包含小提琴與鋼琴的基本演奏方式與特性分析,
以及小提琴與鋼琴的音色比較,
並探討目前在音源分離領域與音樂追蹤領域的研究現況。

第三章、研究方法,
說明本研究細節,如整體的系統架構、音源分離模組設計、音樂追蹤模組設計與改良細節。

第四章、實驗設計與結果,
說明實驗使用的資料集、實驗設計內容以及評估方法,並對於實驗結果進行探討。

第五章、總結,
對於整體研究結果進行總結,提出尚可改進的部分並討論本研究的未來展望。

\pagebreak

\end{document}

