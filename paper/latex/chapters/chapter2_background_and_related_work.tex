\documentclass[class=NCU_thesis, crop=false]{standalone}
\begin{document}

\chapter{背景知識以及文獻回顧}

\section{背景知識}
本研究專注於開發適用於小提琴與鋼琴的系統,
因此本節將介紹小提琴與鋼琴的基本知識與特性。

\subsection{小提琴與鋼琴的演奏特性}
小提琴是一種弓弦樂器,演奏方式透過左手與右手的協調配合來實現。
左手透過指腹按壓琴弦來調整音高,常見的左手指法技巧有滑音、抖音等等。
滑音為通過指腹沿琴弦滑動到不同位置,產生音符連續變化的音色,
抖音則是通過手腕的擺動,使按壓點的觸碰面積改變產生微小的音高波動,
使音符聽起來有抖動的音色;
右手握弓並將弓毛貼住琴弦,透過手臂與手腕的上下來回動作使弓與琴弦摩擦並發出聲音,
常見的弓法技巧有連弓、跳弓等等。
連弓技巧是在一弓之內演奏多個音符,產生連貫流暢的音色。
跳弓則是使弓在琴弦上跳動,產生彈性與清晰的音色。

鋼琴是帶有琴弦的鍵盤樂器,演奏方式透過雙手與雙腳的協調配合來實現。
雙手透過指腹敲擊琴鍵,使琴槌敲擊琴弦發出聲音,
其中可以透過不同的敲擊力度與速度產生不同的音量與音色。
雙腳透過踩踏鋼琴下的踏板,改變聲音的音色。
踏板大致可分為延音踏板與弱音踏板,延音踏板可使音符響起的時間更長,
而弱音踏板則是使彈奏的音符更為柔和。

由於樂器本身的不同,使兩者樂器演奏出來的音樂效果也大不相同,
小提琴透過左右手的控制,相較於鋼琴固定的琴鍵而言,更可以表現出細膩的音色變化,
但對於多聲部或複雜和聲的音樂作品,鋼琴的和聲能力與音域的寬闊更能夠勝任這類的作品。

感覺這兩節可以合起來變一節就好

\subsection{小提琴與鋼琴的音色分析} 
從上一節可以得知,小提琴與鋼琴的聲音是非常不同的,
但我們如何分辨出哪個聲音是小提琴或是鋼琴,
為什麼兩種樂器演奏同一段旋律聽起來會不一樣呢?
這是因為人耳所聽到的聲音可以分為基頻與諧波,
通常基頻代表的是因高,而產生出來的諧波則是表現了聲音的音色,
我們拿{\colorbox{yellow}{圖}}來舉個例子,
此圖為小提琴與鋼琴演奏同一個音高C4三秒鐘的特徵圖,
C4的頻率約為262赫茲,因此可以看到在y=262Hz的特徵很明顯。
而諧波通常出現在整數倍的地方,所以在262*2,262*3也會有明顯的特徵



根據上一節介紹的特性blablabla
介紹小提琴與鋼琴的音色特性
帶出不同人或樂器演奏出的音色都不同、導致音源分離與音樂追蹤的難處

\section{文獻回顧}
\subsection{音源分離相關研究}
引用音源分離的相關論文並進一步討論

\subsection{音樂追蹤相關研究}
DTW、DTW的各種變形、ODTW、RL、
引用音樂追蹤的相關論文並進一步討論

\end{document}