\documentclass[class=NCU_thesis, crop=false]{standalone}
\begin{document}

\chapter{總結}

\section{結論}
本研究提出了一套結合音源分離模組與音樂追蹤模組的小提琴演奏追蹤系統,
此系統能將小提琴與鋼琴的混合音源分離成參考音源與伴奏音源,
並結合現場小提琴音訊實現即時追蹤與伴奏的效果。
在音源分離模組方面,本研究採用Band-Split RNN模型
並提出了簡單的頻帶切割方法,有效提升模型的分離效果。
在音樂追蹤模組方面,本研究改進了音訊特徵擷取方式,
即時處理串流音訊使其他模塊能使用音訊特徵。此外,
我們也改良了音樂偵測模塊,加入了平均振幅閥值判斷法加速過濾靜音片段。
我們也改進了線上動態時間規整演算法(ODTW),
使系統能夠更精確地對齊現場音訊並即時輸出。
最後,改進了貪心向後對齊演算法(GBA),
使模組在決定輸出位置時能參考更精準的資訊。

研究實驗結果顯示,音源分離模型在資料缺乏與資料充足的情況下,
效果皆優於現有的基線模型。
針對加入頻帶切割方法是否對模型有正面影響的實驗結果表明,
對不同樂器使用頻帶切割方法可提升模型效果。
另外,音樂追蹤模組的實驗結果顯示,在隨機速度下的即時追蹤效果最好的延遲為35.81ms,
最差的結果也只延遲了4個16分音符,
即使從慢版速度(80bpm)加速到快板速度(160bpm),即時追蹤效果也同樣穩定。
在不同特徵的即時追蹤效果,雖然比相同特徵的效果要差,
但與離線追蹤效果相比,即時追蹤的最佳路徑幾乎與離線追蹤的最佳路徑重疊,
顯示即時追蹤效果也能達到與離線追蹤相同的穩定度與精準度。

\pagebreak

\section{未來展望}
本研究成功在使用不同特徵且即時的情況下追蹤小提琴演奏的位置並輸出伴奏音訊,
但此系統還有許多可以改進的地方,未來的研究可以探索以下幾個方向。

在音源分離方面,本研究使用了較輕量的Band-Split RNN模型進行訓練,
未來可以嘗試使用更大的模型,例如目前效果最佳的模型Hybrid Transformers~\cite{rouard2023hybrid}。
此模型能同時在時頻域和波形域進行訓練,且不需要訓練多個分離目標模型。
此外,本研究只針對小提琴與鋼琴的混合音源進行研究,
但其實在古典音樂中,還有許多不同的樂器合奏配置。
因此,未來希望能擴展音源分離技術以涵蓋更多種類的古典樂器。

在音樂追蹤方面,由於輸出的位置可能會連續停留在同一位置,
導致伴奏聽起來有失真或雜訊的感覺,未來可以研究輸出音訊的平滑化技術,
或是從輸出位置估計目前的演奏速度來調整整段伴奏的快慢。
另外,可以結合數位音訊工作站(Digital Audio Workstation, DAW),將輸入與輸出皆透過DAW處理,
使程式更簡化,並且能更快速地轉換特徵,從而降低系統延遲。

希望未來能在這些方面進行深入研究,以進一步提升系統的性能和應用範圍。

% 人機互動:研究如何結合人機互動技術,使演奏者能更靈活地控制系統,實現更加個性化的演奏體驗。

% . 將音源分離模型合而為一
% . 拓展不同樂器
% . 使伴奏輸出音訊更平滑
% . 結合數位音訊工作站處理輸入輸出
\end{document}