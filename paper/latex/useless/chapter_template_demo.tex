\documentclass[class=NCU_thesis, crop=false]{standalone}
\usepackage{showexpl}

\begin{document}

\chapter{章名(章節示例)}
章內容內容內容內容內容 \\
內容內容內容

\section{節名}
節內容內容內容內容內容 \\
內容內容內容

\subsection{小節名}
內容內容內容 \\
內容內容內容

\subsubsection{小小節}
內容內容內容 \\
內容內容內容

\paragraph{段}
內容內容內容 \\
內容內容內容

\subparagraph{小段}
內容內容內容 \\
內容內容內容


\chapter{文字}
第一行。
仍是第一行。 \\
第二行。


\chapter{圖片}
\section{插入單一圖片}
\fig[0.15][fig:label_test][!hbt]{logo-Linux.png}[caption][short caption]

\section{插入多張圖片}
\begin{figure}[!hbt]
    %\captionsetup[subfigure]{labelformat=empty} % 完全隱藏圖號
    \centering
    \subcaptionbox
        {caption\_1
        \label{fig:subfig_fig1}}
        {\includegraphics[width=0.3\linewidth]{fig1.png}}
    ~
    \subcaptionbox
        {caption\_2
        \label{fig:subfig_fig2}}
        {\includegraphics[width=0.3\linewidth]{fig2.eps}}
    \vspace{\baselineskip} % 分隔上下列
    \subcaptionbox
        {caption\_3
        \label{fig:subfig_fig3}}
        {\includegraphics[width=0.6\linewidth]{fig3.png}}
    \caption{caption, 使用 \subref{fig:subfig_fig2}取得子圖(Debian)編號 }
    \label{fig:label}
\end{figure}


\chapter{表格}
\section{一般表格}
\begin{table}[h]
    \centering
    \caption{Solution}
    \begin{tabular}{| l | l |}
        \hline
        Component & Concentration(mM) \\ \hline
        \ce{NaCl} & 118.0 \\ \hline
    \end{tabular}
\end{table}

\section{自動折行表格}
\begin{table}[h]
    \centering
    \begin{tabularx}{\textwidth}{| l | X |}
        \hline
        short & short short \\ \hline
        long & long long long long long long long long long long  long long long long long long long long long long\\ \hline
    \end{tabularx}
\end{table}

\end{document}